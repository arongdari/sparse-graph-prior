% !TEX TS-program = pdflatexmk
\documentclass{article} % For LaTeX2e
\usepackage{nips15submit_e,times}
\usepackage{hyperref}
\usepackage{url}
%\documentstyle[nips14submit_09,times,art10]{article} % For LaTeX 2.09

\usepackage{amsthm}
\usepackage{amsmath}
\usepackage{amssymb}
\usepackage{mathtools}
\usepackage{thmtools}

\usepackage{algpseudocode,algorithm,algorithmicx}  

%\newtheorem{theorem}{Theorem}
%\newtheorem{corollary}{Corollary}[theorem]
%\newtheorem{lemma}[theorem]{Lemma}
%\newtheorem{example}{Example}[theorem]

%\theoremstyle{definition}
\declaretheorem[style=definition,qed=$\blacksquare$]{definition}
\declaretheorem[style=definition,qed=$\blacktriangle$,sibling=definition]{example}

\declaretheorem[style=plain,qed=$\blacktriangle$,sibling=definition]{theorem}
\declaretheorem[style=plain,qed=$\lhd$,sibling=definition]{lemma}

%\newtheorem{definition}{Definition}
\newtheorem{assumption}{A}
\newtheorem{question}{Q}

\newcommand\myeq{\stackrel{\mathclap{\tiny\mbox{d}}}{=}}

\newcommand\mc{\mathcal} %calligraphic
\newcommand\ts{\mathcal} %tensor
\newcommand\mt{} %matrix
\newcommand\vt{\mathbf} %vector
\newcommand\fn{} %function
\newcommand\triple[3]{(#1 \stackrel{#2}\rightarrow #3)}
%\newcommand\triple[3]{(#1, #2, #3)}

\title{Collaborative Matrix Completion}

\author{
Dongwoo Kim
}

% The \author macro works with any number of authors. There are two commands
% used to separate the names and addresses of multiple authors: \And and \AND.
%
% Using \And between authors leaves it to \LaTeX{} to determine where to break
% the lines. Using \AND forces a linebreak at that point. So, if \LaTeX{}
% puts 3 of 4 authors names on the first line, and the last on the second
% line, try using \AND instead of \And before the third author name.

\newcommand{\fix}{\marginpar{FIX}}
\newcommand{\new}{\marginpar{NEW}}

\nipsfinalcopy % Uncomment for camera-ready version

\begin{document}

\maketitle

\begin{abstract}
Low-rank or approximately low-rank matrix approximation is a popular method to complete a matrix given partial observation of its entry. As the seminar work by \cite{candes2009exact} pointed out, one can recover the original matrix with high probability given a certain number of samples. In practice, when the number of observations is not satisfying this criteria, one widely used heuristic to overcome the shortage of samples is to jointly completing multiple matrices that shares some common dimensions. For example, to get a reliable estimator of user-movie rating matrix, one may use movie-actor matrix as additional information to get more information about movies. In this work, we would like to uncover when and how this joint completion would work and why.
\end{abstract}

\section{Preliminary}
Assume that unknown rank-$r$ matrix $M$ is $m_1 \times m_2$. The goal of single matrix completion is to recover original matrix $M$ from partially observed entries. The singular value decomposition (SVD) is
\begin{equation}
M = \sum_{i=1}^{r}\sigma_i u_i v_i^\top,
\end{equation}
where $\sigma_i,...,\sigma_r \geq 0$ are the singular values, and $u_1,...,u_r$ and $v_1,...,v_r$ are two sets of orthonormal singular vectors. The degree of freedom of this matrix is $(m_1+m_2)r - r^2$, which means we cannot recover the original matrix if less than $(m_1+m_2)r - r^2$ entries are available.  

Let $\Omega_M$ be a set of indices of observed entries, i.e. $(i,j) \in \Omega_M$ if $M_{ij}$ is observed, and $\mc{P}_{\Omega}: \mathbb{R}^{m_1\times m_2} \rightarrow \mathbb{R}^{m_1\times m_2}$ be the orthogonal projection onto index set $\Omega$ which vanishes outside of $\Omega$; that is $\bar{M} = \mc{P}_{\Omega}(M)$ is defined as
\begin{equation}
\bar{M}_{ij} = \left\{
  \begin{array}{lr}
    M_{ij}, & (i,j) \in \Omega\\
    0, & \text{otherwise}.
  \end{array}
\right.
\end{equation}

The spectral norm of matrix $M$ is denoted by
\begin{equation}
||M|| := \sup_{u\in \mathbb{R}^n:||u||=1}||Mu|| = \sup_{j\in[n]}\sigma_j(M),
\end{equation}
which corresponds to the largest singular value of matrix $M$. The nuclear norm of matrix $M$ is denoted by
\begin{equation}
||M||_* := \sum_{j\in[n]}\sigma_j(M),
\end{equation}
which corresponds to the sum of singular values of matrix $M$. Additionally, we let $||M||_\infty := \max_{i,j}|M_{ij}|$.

The Euclidean inner product between two matrices is defined by
\begin{align}
\langle M, N \rangle := \text{tr}(M^\top N),
\end{align}
and corresponding Frobenius norm is denoted as
\begin{align}
||M||_F
\end{align}

By the duality between nuclear and spectral norm, we can compute an upper bound on the Euclidean inner product between matricies
\begin{align}
\langle M, N \rangle & = ||M|| \langle \frac{M}{||M||}, N \rangle \\
& = ||M|| \text{tr}(\frac{M}{||M||}^\top N)\\
& \leq ||M|| \sup_{||X|| \leq 1} \text{tr}(X^\top A)\\
& = ||M||\,||A||_*.
\end{align}

\subsection{Sampling scheme}
Let $\pi_{ij}$ be the probability to observe the $(i,j)$ entiry of $M$. Let $C_j = \sum_{i=1}^{m_1} \pi_{ij}$ be the probability of observing an entry from column $j$ and $R_i = \sum_{j=1}^{m_2} \pi_{ij}$ be the probability of observing an entry from row $i$.
If there is a noise in observation, then noisy observation of entry
\begin{align}
\bar{M}_{ij} = M_{ij} + \sigma\xi_{ij}, \quad (i,j) \in \Omega
\end{align}
where noise variable $\xi_{ij}$ satisfies $\mathbb{E}[\xi_{ij}] = 0$ and $\mathbb{E}[\xi_{ij}^2] = 1$, and $\sigma$ is the variance of the noise.

In terms of the trace regression model, the observation also can be defined as
\begin{align}
\bar{M}_{i} = \text{tr}(E_i^\top M) + \sigma\xi_i, \quad i=1,...,n
\end{align}
where $E_i$ are random matrices of dimension $m_1 \times m_2$, and tr($X$) denotes the trace of matrix $X$. Each $E_i$ is i.i.d copy from a random matrix $E$ having distribution $\Pi$ on the set $\mathcal{E} = \{e_j(m_1)e_k^\top(m_2), 1\leq j \leq m_1, 1\leq k \leq m_2 \}$, where $e_j(m)$ is $j$th canonical basis vector in $\mathbb{R}^m$, i.e., zero vector except $i$th entry equal to 1. Following two stochastic terms play an important role in the case of noisy observation analysis. For an Rademacher sequence $\{\epsilon_i\}_{i=1}^{n}$, we define
\begin{align}
\Sigma_R = \frac{1}{n}\sum_{i=1}^{n}\epsilon_i E_i \quad \text{and} \quad\Sigma = \frac{\sigma}{n}\sum_{i=1}^{n}\xi_i E_i.
\end{align}
Let $||M||_{L_2(\Pi)}^2 = \mathbb{E}[\langle E,M \rangle)]^2$, where the expectation is taken over ${E}$. If $\pi_{ij} \geq pm_1m_2$ for all entries, then this implies
\begin{align}
\label{eqn:fro_ineq}
||M||_{L_2(\Pi)}^2 \geq \frac{||M||_F^2}{pm_1m_2}.
\end{align}

\subsection{Orthogonal projection}

Set $U := \text{span}(u_1,...,u_r)$ and $V := \text{span}(v_1,...,v_r)$. It is convenient to introduce the orthogonal decomposition $\mathbb{R}^{m_1 \times m_2} = T \oplus T^\perp$ where $T$ is the linear space spanned by elements of the form $u_ky^\top$ and $xv_k^\top$, $1\leq k \leq r$, where $x$ and $y$ are arbitrary vectors, and $T^\perp$ is its orthogonal complement. The orthogonal projection $\mc{P}_T:\mathbb{R}^{m_1 \times m_2} \rightarrow T$ onto $T$ is given by
\begin{equation}
\mc{P}_T(Z) = P_U(Z) + ZP_V - P_UZP_V,
\end{equation}
where $P_U$ and $P_V$ are the orthogonal projections onto $U$ and $V$ respectively. The orthogonal projection onto $T^\perp$ is given by
\begin{equation}
\mc{P}_{T^\perp}(Z) = (I_{m_1} - P_U)Z(I_{m_2} - P_V),
\end{equation}
where $I_m$ denotes the $m \times m$ identity matrix.

\section{Single Matrix Completion}
The noiseless matrix completion was first studied by \cite{candes2009exact}, where the number of samples needed to recover a matrix of rank $r$ exactly is provided under some incoherent assumptions on the singular vectors of the matrix.
\begin{definition}[Coherence \cite{candes2009exact}]
Let $U$ be a subspace of $\mathbb{R}^n$ of dimension $r$, i.e., $U := \text{span}(u_1, ..., u_r)$, and $P_U$ be the orthogonal projection onto $U$. Then the coherence of $U$ is defined to be
\begin{equation}
\mu(U) := \frac{m_1}{r}\max_{1\leq i \leq m_1}||P_U e_i||^2,
\end{equation}
where $e_i$ is the $i$th canonical basis vector.
\end{definition}

\begin{assumption}
The coherences obey $\max(\mu(U),\mu(V)) \leq \mu_0$ for some positive $\mu_0$.
\end{assumption}
\begin{assumption}
The matrix $E=\sum_{1\leq k \leq r} u_k v_k^\top$ has a maximum entry bounded by $\mu_1 \sqrt{r/(m_1m_2)}$ in absolute value for some positive $\mu_1$.
\end{assumption}

\begin{theorem}[\cite{candes2009exact}]
Under the assumption 1 and 2, a nuclear norm minimisation
\begin{align}
\text{minimise}&\quad ||X||_* \\
\text{subject to}&\quad \mc{P}_{\Omega}(X) = \mc{P}_{\Omega}(M) 
\end{align}
perfectly recovers the original matrix with high probability if the number of uniformly sampled entires $n$ obey $n\geq \mathcal{O}(\max(\mu_1^2,\mu_0^{1/2},\mu_1,\mu_0n^{1/4}) m_2 r \beta \log m_2)$ for some $\beta > 2$.
\end{theorem}
Later, a tighter analysis of the same convex relaxation was developed in \cite{candes2010power,recht2011simpler}. More practical settings, where the few observed entries are corrupted by noise, has been extensively studied recently \cite{candes2010matrix,keshavan2010matrix,negahban2012restricted,klopp2014noisy,lafond2015low}. 
%These studies show that when the distribution on noise is additive and sub-exponential, then the prediction error with the nuclear norm minimiser $\hat{X}$ satisfies with high probability
%\begin{equation}
%\frac{||\hat{X} - X||^2_{F}}{m_1 m_2} = \mathcal{O}\bigg(
%\frac{(m_1+m_2) \text{rank}(X) \log (m_1 + m_2)}{m}
%\bigg),
%\end{equation}
%where $||\cdot||_F$ denotes the Frobenius norm, and $X\in\mathbb{R}^{m_1 \times m_2}$.
To prevent an overfitting of the noisy matrix observation, a penalised nuclear norm estimator of $M$ has been introduced
\begin{align}\label{eqn:nn_est}
\hat{X} = {\arg\min}_{||{A}||_\infty \leq \gamma} \bigg\{\frac{1}{n} \sum_{(i,j) \in \Omega}(\bar{M}_{ij} - {A}_{ij})^2 + \lambda ||{A}||_* \bigg\},
\end{align}
where $\bar{M}_{ij}$ is a noisy observation of $M_{ij}$, $\lambda > 0$ is a regularisation parameter, and $\gamma$ is an upper on the absolute value of each entry. The optimiser consists of the sum of a data fitting terms and a nuclear norm penalisation term.

\begin{theorem}[Noisy matrix completion \cite{klopp2014noisy}] 
Let noisy observation of $M$ be sampled i.i.d. from distribution $\Pi$ with known variance of the noise $\sigma$. If $\lambda \geq 3||\Sigma||$ and following two assumptions are satisfied:

\begin{assumption}
\label{assume:row_column}
There exists a positive constant $L \geq 1 $ such that
$\max_{i,j}(C_i, R_j) \leq L / \min(m_1, m_2)$.
\end{assumption}

\begin{assumption}
\label{assume:min_p}
There exists a positive constant $p \geq 1 $ such that
$\pi_{jk} \geq (p m_1 m_2)^{-1}$.
\end{assumption}

Then, there exist numerical constants $c_1$, and $c_2$ such that
\begin{equation}
\frac{||\hat{X} - M||_F^2}{m_1m_2} \leq \max\bigg\{ c_1 p^2 m_1 m_2 r (\lambda^2 + \gamma^2(\mathbb{E}||\Sigma_R||)^2), c_2\gamma^2p\sqrt{\frac{\log(m_1 + m_2)}{n}} \bigg\},
\end{equation}
with probability at least $1 - 2/(m_1 + m_2)$.
\end{theorem}

An upper bound of the stochastic term $\mathbb{E}||\Sigma_R||$ could be also obtained under certain assumptions (e.g., sub-exponential noise). See more details in \cite{klopp2014noisy}.

%\textbf{Sketch of Proof}. From the definition of estimator in Equation $\ref{eqn:nn_est}$, we know that
%\begin{equation}
%\frac{1}{n} \sum_{(i,j) \in \Omega}(\bar{M}_{ij} - \hat{X}_{ij})^2 + \lambda||\hat{X}||_*
%\leq \frac{1}{n} \sum_{(i,j) \in \Omega}(\bar{M}_{ij} - {M}_{ij})^2 + \lambda||M||_*,
%\end{equation}
%which implies
%\begin{equation}
%\frac{1}{n}||\mathcal{P}_\Omega(M - \hat{X}) ||_F^2 \leq 2||\Sigma|| ||M - \hat{X}||_* + \lambda(||M||_* - ||\hat{X}||_*).
%\end{equation}

\section{Joint Matrix Completion}
Let $A_0 \in \mathbb{R}^{m_1 \times m_2}$ and $B_0 \in \mathbb{R}^{m_1 \times m_3}$ be two matrices where the first dimension represents the same object, i.e. user-rating matrix $A$ and user-attribute matrix $B$. When the number of observed entries $n_A$ of matrix $A_0$ is insufficient to obtain a stable estimator of the unobserved entries, one widely used heuristic method is a joint factorisation of $A_0$ and $B_0$ with the hope that the both matrices share some common low-rank structure in its first dimension. The collaborative matrix factorisation has been widely used in practice, but there is not theoretical guarantee so far.

One of our main question is how many samples are required to perfectly recover the original matrices $A_0$ and $B_0$ under what assumptions. More precisely, an interesting question might be the number of observation $n_A$ of matrix $A_0$ needed to recover the matrix $A_0$ given the number of observation $n_B$ of matrix $B_0$ or vice versa. Often, it is more easier to obtain the entries of $B_0$ instead of those in $A_0$, therefore, understanding the nature of joint completion will provide some guidelines to the joint completion approach. We start from an assumption to make the analysis feasible.

\begin{assumption}\label{assume:share}
Let SVD of $A_0 = \sum_{i=1}^{r_A} \sigma_i^{(A_0)} u_i^{(A_0)} v_i^\top$ and $B_0 = \sum_{i=1}^{r_B} \sigma_i^{(B_0)} u_i^{(B_0)} w_i^\top$. We assume that the rank of two matrices are the same, i.e., $r_M = r_N = r$, and the singular vector $u_i^{(A_0)}$ and $u_i^{(B_0)}$ are the same in order to share the common low-rank structure between two matrices\footnote{Remark: most of existing knowledge graph factorisation models such as RESCAL follow this assumption if we consider each relation as a matrix (a knowledge graph is a collection of matrices).}.
\end{assumption}

Let $Z_0 = [A_0, B_0] \in \mathbb{R}^{m_1 \times (m_2+m_3)}$ be the horizontally combined matrix of $A_0$ and $B_0$. The SVD of $Z$ is $\sum_{i=1}^{r}\sigma_i u_i v_i^\top$ where $v_i^\top = [v_i^{(A_0)\top}, v_i^{(B_0)\top}]$ is the stacked right singular vectors of $A_0$ and $B_0$. If both singular vectors satisfy the incoherence assumptions, the joint matrix completion problem is reduced to the single matrix completion problem with different sampling rates between the first $m_2$ columns and the rightmost $m_3$ columns of combined matrix. Some of the previous studies focus on a non-uniform sampling distribution of matrix \cite{foygel2011learning,lounici2011optimal,negahban2012restricted,klopp2014noisy}. For example, popular movies are more likely to be rated by many users in a collaborative filtering. This will lead to different sampling distributions between different columns (rows). 

Our problem is a special case where we sample $n_M$ samples from leftmost $m_2$ columns and $n_N$ from rightmost $m_3$ columns. More formally, let $\Pi_A$ be a distribution on set $\mc{E}^{A} = \{e_j(m_1)e_k^\top(m_2+m_3), 1\leq j \leq m_1, 1\leq k \leq m_2\}$, and $\Pi_{B}$ a distribution on set $\mc{E}^{B} = \{e_j(m_1)e_k^\top(m_2+m_3), 1\leq j \leq m_1, 1+m_2\leq k \leq m_2+m_3\}$. And define $\Pi_\rho = \rho \Pi^{A} + (1-\rho)\Pi^{B}$ where $0 < \rho < 1/2$. If $\Phi_A$ and $\Phi_B$ satisfy Assumption \ref{assume:min_p} with parameter $p_A$ and $p_B$, Equation \ref{eqn:fro_ineq} implies
\begin{align}
||Z_0||_{L_2(\Pi_\rho)}^2 \geq \min(\frac{\rho}{p_Am_1m_2},\frac{1-\rho}{p_Bm_1m_3})||Z_0||_F^2 \geq \frac{\rho}{pm_1(m_2+m_3)},
\end{align}
where $p = \min(p_A, p_B)$.

%And define $\Pi_\rho = \rho \Pi^{(M)} + (1-\rho) \Pi^{(N)}$ where $0 \leq \rho \leq 1$. %If we set $\rho = \frac{n_M}{n_M+n_N}$, and 
%There are $p^{(M)}$ and $p^{(N)}$ which satisfies Assumption \ref{assume:min_p} for each distribution, respectively, then the inequality in Equation \ref{eqn:fro_ineq} also holds as well
%\begin{align}
%||Z||_{\Pi_\rho}^2 \geq \frac{||Z||_F^2}{p'm_1(m_2+m_3)},
%\end{align}
%where $p' = \min(\rho p^{(M)}m_1m_2, (1-\rho) p^{(N)}m_1m_3)$.

\begin{question}[$n_A$ given $n_B$]
Given $n_B$ uniformly sampled entries of matrix $B_0$, how many samples do we need from $A_0$ to perfectly recover matrix $A_0$? More practically, what will be the relation between the reconstruction error on $A_0$ and the sample size $n_A$ and $n_B$.
\end{question}
Let the penalised estimator of joint matrix $Z$
\begin{align}
\label{eqn:z_estimator}
\hat{Z} = {\arg\min}_{||{A}||_\infty \leq \gamma} \bigg\{\frac{1}{n_M + n_N} \sum_{(i,j) \in \Omega}(\bar{Z}_{ij} - {A}_{ij})^2 + \lambda ||{A}||_* \bigg\},
\end{align}
where $\bar{Z}_{ij}$ denotes the noisy observation of $(i,j)$ entry of $Z_0$. Let $\hat{A}$ (\textit{resp.} $\hat{B}$) be the corresponding part of $A_0$ (\textit{resp.} $B_0$) in $\hat{Z}$, then the first question corresponds to identify an upper bound on
$||\hat{A} - A_0||_F^2$
with respect to the other quantities such as $n_A$ and $n_B$.

The Frobenius norm error of the combined matrix can be decomposed into two different parts:
\begin{equation}
\label{eqn:decomp}
\frac{||\hat{Z} - Z||_F^2}{m_1(m_2+m_3)} = \frac{||\hat{A} - A_0||_F^2}{m_1(m_2+m_3)} + \frac{||\hat{B} - B_0||_F^2}{m_1(m_2+m_3)}.
\end{equation}

%From the previous work, it is possible to obtain an upper bound on Frobenius error of combined matrix. However, this does not give any information gain obtained from the shared low rank structure. Therefore, a lower bound on the second term in RHS of equation \ref{eqn:decomp} is needed to induce an upper bound on equation \ref{eqn:upper_m}.

From the definition of estimator in Equation \ref{eqn:z_estimator}, the following inequality holds (the objective function  w.r.t optimiser $\hat{Z}$ is always less or equal than the objective function w.r.t unknown true matrix $Z_0$).
\begin{align}
\frac{1}{n} \sum_{(i,j) \in \Omega}(\bar{Z}_{ij} - \hat{Z}_{ij})^2 + \lambda ||\hat{Z}||_* 
\leq \frac{1}{n} \sum_{(i,j) \in \Omega}(\bar{Z}_{ij} - {Z}_{0ij})^2 + \lambda ||Z_0||_*,
\end{align}
where $n = n_M + n_N$. This implies
\begin{equation}
\frac{1}{n}||\mathcal{P}_\Omega(Z_0 - \hat{Z}) ||_F^2 \leq 2||\Sigma||\,||Z_0 - \hat{Z}||_* + \lambda(||Z_0||_* - ||\hat{Z}||_*). \\
\end{equation}
Applying Lemma \ref{lmm:normdiff} and using triangular inequality $||Z_0 - \hat{Z}||_* \leq ||\mc{P}_{Z_0}(Z_0 - \hat{Z})||_* + ||\mc{P}_{Z_0}^\perp(Z_0 - \hat{Z})||_*$, we obtain
\begin{equation}
\frac{1}{n}||\mathcal{P}_\Omega(Z_0 - \hat{Z}) ||_F^2 = \frac{1}{n}||\mc{P}_{\Omega_A}(A_0-\hat{A})||_F^2 + \frac{1}{n}||\mc{P}_{\Omega_B}(B_0-\hat{B})||_F^2
\leq \frac{5}{3}\lambda ||\mc{P}_Z(Z_0-\hat{Z})||_*.
\label{eqn:tri}
\end{equation}

%Let $\alpha = ||M - \hat{M}||_\infty (\leq 2\gamma)$. If $\alpha^{-1}(M-\hat{M}) \in \mc{C}(t)$, then by Lemma \ref{lmm:constraintset}
%\begin{equation}
%\frac{n_M}{2n\alpha^2}||M-\hat{M}||_{\Pi^{(M)}}^2  - 44 \frac{ptm_1m_2n_M}{n}(\mathbb{E}||\Sigma_R^{(M)}||)^2 
%\leq \frac{n_M}{n}\frac{1}{\alpha^2n_M}||\mc{P}_{\Omega_M}(M-\hat{M})||_F^2 \label{eqn:apl_lmm}
%\end{equation}
%Same argument can be made on the lower bound of $||\mc{P}_{\Omega_N}(N-\hat{N})||_F^2$ if $\beta^{-1}(N-\hat{N}) \in \mc{C}(t)$
%\begin{equation}
%\label{eqn:apl_lmm2}
%\frac{n_N}{2n\beta^2}||N-\hat{N}||_{\Pi^{(N)}}^2  - 44 \frac{ptm_1m_3n_N}{n}(\mathbb{E}||\Sigma_R^{(N)}||)^2 
%\leq \frac{n_N}{n}\frac{1}{\beta^2n_N}||\mc{P}_{\Omega_N}(N-\hat{N})||_F^2 
%\end{equation}
%Combining \ref{eqn:tri}, \ref{eqn:apl_lmm} and \ref{eqn:apl_lmm2} yields
%\begin{align}
%&\frac{n_M}{2n}||M-\hat{M}||_{\Pi^{(M)}}^2 + \frac{n_N}{2n}||N-\hat{N}||_{\Pi^{(N)}}^2&  \notag\\
%&\quad \leq \frac{5}{3}\lambda ||\mc{P}_Z(Z-\hat{Z})||_* + 176 \frac{ptm_1\gamma^2}{n}\bigg(m_2n_M(\mathbb{E}||\Sigma_R^{(M)}||)^2 + m_3n_N(\mathbb{E}||\Sigma_R^{(N)}||)^2\bigg).&
%\end{align}
%Equation \ref{eqn:fro_ineq} implies
%\begin{align}
%&\frac{n_M}{2nm_1m_2p}||M-\hat{M}||_F^2 + \frac{n_N}{2nm_1m_3p}||N-\hat{N}||_F^2& \notag\\
%&\quad \leq \frac{5}{3}\lambda ||\mc{P}_Z(Z-\hat{Z})||_* + 176 \frac{ptm_1\gamma^2}{n}\bigg(m_2n_M(\mathbb{E}||\Sigma_R^{(M)}||)^2 + m_3n_N(\mathbb{E}||\Sigma_R^{(N)}||)^2\bigg).&
%\end{align}
%The upper bound on $||\mc{P}_Z(Z-\hat{Z})||_*$ can be decomposed into Frobenius error on two sub matrices:
%\begin{align}
%||\mc{P}_Z(Z-\hat{Z})||_* & \leq \sqrt{2r}||Z-\hat{Z}||_F \\
%& =  \sqrt{2r} \sqrt{||M-\hat{M}||_F^2 + ||N-\hat{N}||_F^2}\\
%& \leq 2r + \frac{1}{4}\bigg(||M-\hat{M}||_F^2 + ||N-\hat{N}||_F^2\bigg),
%\end{align}
%where the first inequality comes from $||A||_* \leq \sqrt{\text{rank}(A)}||A||_F$ and $\text{rank}(\mc{P}_Z(A)) \leq \text{rank}(P_{U^\perp}AP_V) + \text{rank}(P_{U}A)$, and we use the inequality $ab \leq a^2 + b^2/4$ in the third line.


\begin{lemma} \label{lmm:constraintset} Given constraint set
\begin{equation}
\mc{C}(t) = \bigg\{ A \in \mathbb{R}^{m_1 \times m_2}: ||A||_\infty = 1, ||A||_{L_2(\Pi)}^2 \geq \sqrt{\frac{64\log(m_1+m_2)}{\log(6/5)n}}, ||A||_* \leq \sqrt{t}||A||_F \bigg\},
\end{equation}
%where $A$ is a submatrix of $B$. Note that the condition $||A||_* \leq \sqrt{t}||B||_F$ always satisfied if $t$ = rank($B$) since $||A||_* \leq ||B||_*$. 
Let $E_i$ be i.i.d. from distribution $\Pi$ on $\mc{E}$ which satisfies Assumptions \ref{assume:row_column} and \ref{assume:min_p}, and $\Sigma_R = n^{-1}\sum_{i=1}^{n}\epsilon_i E_i$, where $\{\epsilon_i\}_{i=1}^{n}$ is Rademacher sequence. Then, for all $A\in \mc{C}(t)$
\begin{align}
\frac{1}{2}||A||_{L_2(\Pi)}^2  - 44 ptm_1m_2(\mathbb{E}[||\Sigma_R||])^2 \leq \frac{1}{n}||\mc{P}_{\Omega}(A)||_F^2
\end{align}
with probability at least $1-2(m_1+m_2)^{-1}$.
\end{lemma}

\begin{lemma}\label{lmm:normdiff}For any pair of matrices $A, B \in \mathbb{R}^{m_1 \times m_2}$, we have
\begin{equation}
||A||_* - ||B||_* \leq ||\mc{P}_A(A-B)||_* - ||\mc{P}_A^\perp(A-B)||_*
\end{equation}
\end{lemma}
\textbf{Proof} The definition on $\mc{P}_A^\perp(B)$ for any matrix $B$ implies $||A + \mc{P}_A^\perp(B-A)||_* = ||A||_* + ||\mc{P}_A^\perp(B-A)||_*$. Then the nuclear norm on $B$
\begin{align}
||B||_* = ||B + A - A||_* &= ||A + \mc{P}^\perp_A(B-A) + \mc{P}_A(B-A)||_*\\
&\geq ||A + \mc{P}^\perp_A(B-A)||_* + ||\mc{P}_A(B-A)||_*\\
&= ||A||_* + ||\mc{P}^\perp_A(B-A)||_* + ||\mc{P}_A(B-A)||_*,
\end{align}
where we use the triangular inequality on the projected space.

%$A, B \in \mathbb{R}^{m_1 \times m_2}, ||A + \mc{P}_{A}^{\perp}(B)||_* = ||A||_* + ||\mc{P}_{X}^\perp(B)||_*$

\subsection{Lower Bound on $||B_0-\hat{B}||_F^2$}

It is crucial to derive a lower bound on $||N-\hat{N}||_F^2$. We can use the same argument used in \cite{lafond2015low} to show the lower bound.


\begin{question}[Approximate singular vectors]
What if the left singular vectors of $M$ and those of $N$ are not equal, but approximate one another? (i.e., $u_i^{(M)} \approx u_i^{(N)}$)
\end{question}

\begin{question}[Lower bound on $n_M$]
Assume that we have fully observed matrix $N$. How many samples do we need from $M$ to perfectly recover matrix $M$?
\end{question}

\begin{question}[Active learning on $N$ to recover $M$]
In many cases, it is more easy to obtain entries of $N$ instead of $M$. The question is which entry of $N$ would be actively sampled to improve the reconstruction performance of $M$.
\end{question}


%If the low-rank representations are the same, then we can apply the standard theories on the combined matrix $[X, Y] \in \mathbb{R}^{m_1 \times (m_2 + m_3)}$. However, this is not the case, under which conditions we can apply joint optimisation for reconstructing $X$ and what would be the error bound of the estimate? Following questions might be of interest to solve this problem.
%\begin{itemize}
%\item What are the corresponding optimiser of the nuclear norm optimiser for the joint factorisation?
%\item What structural assumptions should be made to link two matrices? for example, a certain similarity assumption between two singular value vectors or two low-rank representations? 
%\item Noise / noiseless assumptions. which noise model?
%\item Sampling distribution. uniform? non-uniform?
%\end{itemize}

\bibliographystyle{apalike}
\bibliography{ref}

\end{document}
