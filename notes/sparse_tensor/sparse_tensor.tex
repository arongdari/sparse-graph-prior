% !TEX TS-program = pdflatexmk
\documentclass{article}

\usepackage{graphicx}
\usepackage{amssymb}
\usepackage{amsmath}
\usepackage[a4paper, total={6in, 8in}]{geometry}
\usepackage{amsthm}
\usepackage{natbib}


\newtheorem{theorem}{Theorem}[section]
\newtheorem{corollary}{Corollary}[theorem]
\newtheorem{lemma}[theorem]{Lemma}

\title{Sparse Graph Prior for Knowledge Graph}
\date{\today}
\author{Dongwoo Kim\\ANU}

\begin{document}

\maketitle

\section{Completely Random Measure}
A completely random measure (CRM) $\mu$ on $\mathbb{R}_+$ is a random measure such that for any countable number of disjoint measurable sets $A_1, A_2, ...$ of $\mathbb{R}_+$, the random variable $\mu(A_1), \mu(A_2), ...$ are independent and $\mu(\cup_i A_i) = \sum_i \mu(A_i)$. If one assumes that the distribution of $\mu([t,s])$ only depends on $t-s$ then the CRM takes the form of $\mu = \sum_{i=1}^{\infty}w_i\delta_{\theta_i}$ where $(w_i, \theta_i)$ are the points of a Poisson point process on $\mathbb{R}_+^2$ with L\'{e}vy intensity measure $\nu(dw, d\theta) = \rho(dw)\lambda(d\theta)$. The Laplace transform of $\mu(A)$ on any measurable set $A$ has a following representation: $\mathbb{E}[e^{-t\mu(A)}] = \exp(-\int_{\mathbb{R}_+ \times A}(1-e^{-tw})\rho(dw)\lambda(d\theta))$ for any $t>0$ and $\rho$ such that $\int_{\mathbb{R}_+}(1-e^{-w})\rho(dw) < \infty$. Laplace exponent is $\psi(t) = \int_{\mathbb{R}} (1 - e^{-t w}) \rho(dw)$.

\section{Caron and Fox Model}

\cite{Caron2015} propose a simple point process on $\mathbb{R}^2$ as a product measure of a complete random measure. They propose a hierarchical model for undirected graphs
\begin{align}
\mu &= \sum_{i=1}^{\infty} w_i \delta_{\theta_i} & &\mu \sim \text{CRM}(\rho, \lambda)\\
D &= \sum_{i,j} n_{ij} \delta_{(\theta_i, \theta_j)} & &D|\mu \sim \text{PP}(\mu \times \mu)\\
Z &=\sum_{i,j} \min(n_{ij} + n_{ji}, 1)\delta_{(\theta_i, \theta_j)}, \label{eqn:cnf}&&
\end{align}
with intensity measure $\nu$ factorising as $\nu(dw, d\theta) = \rho(dw) \lambda(d\theta)$ for a jump part of the measure $\rho$ and Lebesgue measure $\lambda$. $D$ is simply generated from a Poisson process with a product measure as an intensity and can be interpreted as a directed multi-graph.
Given $\mu$, we can directly specify the undirected graph $Z$ as
\[ Pr(z_{ij}=1|w) = 
  \begin{cases}
    1 - \exp(-2w_iw_j)       & \quad i \neq j\\
    1 - \exp(-w_i^2) & \quad i = j.\\
  \end{cases}
\]

They show that the resulting graph is sparse, i.e. \# of edges = $o$(\# of nodes$^2$)\footnote{only counts the nodes which has at least one edge}, if the intensity measure\footnote{This is the L\'{e}vy intensity of the generalised gamma process} is
\begin{align}
\rho(dw) = \frac{1}{\Gamma(1-\sigma)}w^{-1-\sigma}e^{-\tau w}dw,
\end{align}
where the two parameters range
\begin{align}
(\sigma, \tau) \in (0,1) \times [0, +\infty)
\end{align}
and dense if the intensity measure is finite activity, i.e. $\int_{0}^{\infty} \rho(w)dw < \infty$.

The general construction of the sparse graph in Equation \ref{eqn:cnf} results an infinite number of edges due to $\mu(\mathbb{R}_+) = \infty$. A restriction of Lebesgue measure $\lambda$ on $[0, \alpha]$ is used to obtain a finite graph ($\lambda_\alpha = \lambda\delta_{[0, \alpha]}$). Therefore, restricted graph $Z_\alpha$ is defined on the box $[0,\alpha]^2$. We also denote the total mass on $[0, \alpha]^2$ by $Z_\alpha^* = Z_\alpha([0, \alpha]^2)$, and similarly for $D_\alpha^*$ and $\mu_\alpha^*$.

\section{Sparse Prior for Knowledge Graph}
A knowledge base consists of a set of triples (entity, entity, relation) such as (BarackObama, bornIn, Hawaii). The set of triples can be represented as a binary-valued three-way tensor where three dimensions represent entity, entity, and relation, respectively. Here, we directly extend the Caron and Fox's model for the three-way tensor based on two independent completely random measures.
\begin{align}
\mu &= \sum_{i=1}^{\infty} w_i \delta_{\theta_i} & &\mu \sim \text{CRM}(\rho, \lambda)\\
\mu' &= \sum_{k=1}^{\infty} w_k \delta_{\theta_k'} & &\mu' \sim \text{CRM}(\rho', \lambda)\\
D &= \sum_{i,j,k} n_{ijk} \delta_{(\theta_i, \theta_j, \theta_k')}& &D \sim \text{PP}(\mu \times \mu \times \mu') && \\
Z &=\sum_{i,j,k} \min(n_{ijk}, 1)\delta_{(\theta_i, \theta_j, \theta_k')}, &&
\end{align}
where $Z$ is asymmetric in $i$ and $j$ since the knowledge graph is a directed multi-graph. As done in the original model, we can also specify $Z$ as
\[ Pr(z_{ijk}=1|w, w') = 
  \begin{cases}
    1 - \exp(-w_iw_jw_k')       & \quad i \neq j\\
    1 - \exp(-w_i^2w_k') & \quad i = j.\\
  \end{cases}
\]
If we consider $\theta_i$, $\theta_j$, and $\theta_k'$ as nodes in the graph, the above construction will generate a hypergraph where each edge connects three nodes. In the notion of knowledge graphs, it is more intuitive to consider a relation as a type of edge between two entities. In this case, we define two random measures on $\mathbb{R}_+^2$:
\begin{align}
\bar{D} &= \sum_{i,j}\sum_{k}z_{ijk}\delta_{\theta_i,\theta_j}\\
\bar{Z} &= \sum_{i,j}\min(\bar{D}(\{\theta_i, \theta_j\}), 1) \delta_{(\theta_i,\theta_j)},
\end{align}
where $\bar{D}$ is a multigraph, and $\bar{Z}$ is a binary graph of a knowledge base.
\[ Pr(\bar{z}_{ij}=1|w, w') = 
  \begin{cases}
    1 - \exp(-w_iw_j\sum_{k}w_k')       & \quad i \neq j\\
    1 - \exp(-w_i^2\sum_{k}w_k') & \quad i = j.\\
  \end{cases}
\]
To obtain a finite hypergraph (the number of edges is finite), we consider restrictions ${D}_{\alpha\beta}$ and ${Z}_{\alpha\beta}$ to the box $[0,\alpha]^2\times[0,\beta]$. We denote by $Z_{\alpha\beta}^* = Z_{\alpha\beta}([0,\alpha]^2\times[0,\beta])$ the total mass on the restricted area, and similar for $D_{\alpha\beta}^*$ and $\mu_{\alpha}^*$.

\subsection{Generative Process through Urn approach}
Given restriction $\alpha$ and $\beta$, the generative process of $D_{\alpha\beta}$ can be specified as follows:
\begin{enumerate}
\item $\mu_\alpha \sim \text{CRM}(\rho, \lambda_\alpha)$
\item $\mu_\beta' \sim \text{CRM}(\rho', \lambda_\beta)$
\item $D_{\alpha\beta}^* | \mu_\alpha, \mu_\beta' \sim \text{Poisson}(\mu_\alpha^{*2}{\mu'}_\beta^{*})$
\item For $d=1,...,D_{\alpha\beta}^*$:
\begin{enumerate}
\item $\theta_{di} \sim \frac{\mu_\alpha}{\mu_\alpha^*}$
\item $\theta_{dj} \sim \frac{\mu_\alpha}{\mu_\alpha^*}$
\item $\theta_{dk}' \sim \frac{\mu_\beta}{{\mu'}_\beta^{*}}$
\end{enumerate}
\item $D_{\alpha\beta} = \sum_{d=1}^{D_{\alpha\beta}^*} \delta_{(\theta_{di}, \theta_{dj}, \theta_{dk})}$,
\end{enumerate}
where we have used that the total mass of $D_{\alpha\beta}^*$ follows the Poisson distribution. Each node $\theta_i$ is drawn from the normalised CRM (NRM), $\frac{\mu_\alpha}{\mu_\alpha^*}$, which is discrete with probability 1. However, it is not possible to sample $\mu_\alpha$ and $\mu'_\beta$ since these measures have infinite number of atoms. Instead we can simulate finite-dimensional generative process through the urn formulation. Let $\theta_1, ..., \theta_n$ drawn from the normalised CRM $\frac{\mu_\alpha}{\mu_\alpha^*}$. Since NRM is discrete, variables $\theta_1, ..., \theta_n$ takes $l \leq n$ distinct values $\phi_l$, and $m_l$ is the number of variables corresponding to $\phi_l$.
Given total mass $\mu_\alpha^*$ and $\theta_1, ..., \theta_n$, the conditional distribution of $\theta_{n+1}$ can be modelled in terms of exchangeable partition probability function (EPPF):
\begin{align}
\label{eqn:eppf}
\theta_{n+1} | \mu_\alpha^*, \theta_1,...,\theta_n \sim \frac{\Pi_{n+1}^{l+1}(m_1, ..., m_l, 1 | \mu_\alpha^*)}{\Pi_{n}^{l}(m_1, ..., m_l | \mu_\alpha^*)} \frac{1}{\alpha} \lambda_\alpha
+ \sum_{i=1}^{l}\frac{\Pi_{n+1}^{l}(m_1, ..., m_{i}+1, ..., m_l | \mu_\alpha^*)}{\Pi_{n}^{l}(m_1, ..., m_l| \mu_\alpha^*)} \delta_{\phi_l}
\end{align}
where
\begin{align}
\Pi_{n}^l(m_1, ..., m_l|\mu_\alpha^*) = \frac{\sigma^l \mu_\alpha^{*-n}}{\Gamma(n-l\sigma)g_{\sigma}(\mu_\alpha^*)} \int_{0}^{\mu_\alpha^*}s^{n-l\sigma-1}g_{\sigma}(\mu_\alpha^*-s)ds \bigg(\prod_{i=1}^{l} \frac{\Gamma(m_i-\sigma)}{\Gamma(1-\sigma)} \bigg),
\end{align}
and $g_\sigma$ is the pdf of the positive stable distribution.
Finally, the total mass of $\mu_\alpha^*$ and ${\mu'}_\beta^{*}$ follows an exponentially tilted stable distribution where the exact sampler exists \citep{devroye2009random,hofert2011sampling}. 

Using this urn representation, we can rewrite the generative process as
\begin{enumerate}
\item $\mu_\alpha^* \sim P_{\mu_\alpha^*}$
\item ${\mu'}_\beta^{*} \sim P_{{\mu'}_\beta^{*}}$
\item $D_{\alpha\beta}^* | \mu_\alpha, \mu_\beta' \sim \text{Poisson}(\mu_\alpha^{*2}{\mu'}_\beta^{*})$
\item For $d=1,...,D_{\alpha\beta}^*$:
\begin{enumerate}
\item Sample $\theta_{di}$, $\theta_{dj}$, and $\theta_{dk}'$ with Urn process in Eqn \ref{eqn:eppf}
\end{enumerate}
\item $D_{\alpha\beta} = \sum_{d=1}^{D_{\alpha\beta}^*} \delta_{(\theta_{di}, \theta_{dj}, \theta_{dk})}$,
\end{enumerate}

\subsection{Characteristics of Random Graph}
Unlike the two-dimensional space case, graph prior on three dimensional space may have various definition of the sparsity of the graph. Since our focus here is the growth rate of various statistics as the graph restriction $\alpha$ and $\beta$ increase. We list some statistics to characterise the random graph.

To compute the (expected) growth rate of the number of triples with respect to the number of entities and relations, we first identify the growth rate of the number of triples, entities, and relations with respect to the varying graph restriction $\alpha$ and $\beta$. Let $N^{e}_{\alpha\beta}$ be the number of triples given $\alpha$ and $\beta$, $N_{\alpha}$ be the number of entities given $\alpha$, and $N_{\beta}$ be the number of relations given $\beta$. Here are some statistics that might help to shape these characteristics.

\begin{align}
\frac{E[N^e_{\alpha\beta}]}{E[N_\alpha N_\beta]} &\text{ as }\alpha \rightarrow \infty, \beta \rightarrow \infty \\
\frac{E[N_\alpha|\beta]}{\alpha} &\text{ as } \alpha \rightarrow \infty \\
\frac{E[N_\alpha]}{\alpha} &\text{ as } \alpha \rightarrow \infty \\
\frac{E[N_\beta|\alpha]}{\beta} &\text{ as } \beta \rightarrow \infty \\
\frac{E[N_\beta]}{\beta} &\text{ as } \beta \rightarrow \infty \\
\frac{E[N^e_{\alpha\beta}]}{\alpha\beta} &\text{ as }\alpha \rightarrow \infty, \beta \rightarrow \infty \\
\frac{E[N^e_{\alpha}|\beta]}{\alpha} &\text{ as }\alpha \rightarrow \infty
\end{align}


\begin{theorem} \label{thm:edge} Consider the point process $\bar{Z}$ with infinite-activity intensity measures $\rho(dw)$ and $\rho'(dw')$. Given $\mu'$ from $\rho'(dw')$, the number of edges in $\bar{Z}_{\alpha}$ grows quadratically as $\alpha \rightarrow \infty$ almost surely.
\end{theorem}
\begin{proof}
$\sum_{k=1}^{\infty} w_k' < \infty$ a.s. When $\mu'$ is given and the sum of $w_k'$ is finite a.s., we can use the same proof technique used in \cite{Caron2015}.
\end{proof}
What if $\mu'$ is not given? Let $(X_i)$ and $(Y_k)$ be i.i.d. real-valued random variable from $p$ and $q$, respectively, and let $h(x_1, x_2, y_1)$ be a measurable function symmetric in the first two arugments. 
\begin{align}
\frac{2 \sum_{i<j}\sum_{k} h(X_i, X_j, Y_k)}{n_x(n_x -1) n_y} \xrightarrow[]{?} \mathbb{E}[h(X_i, X_j, Y_k)]\quad a.s.\quad as \quad n\rightarrow \infty
\end{align}
If this strong law of the large numbers for two samples is correct, we may proof Theorem 3.1 in more general case ($\mu'$ is not given).

\begin{theorem} Consider the point process $\bar{Z}$ with infinite-activity intensity measures $\rho(dw)$ and $\rho'(dw')$. Let $N_\alpha$ be a number of nodes having at least one connection. Given $\mu'$ from $\rho'(dw')$, the number of nodes $N_\alpha$ in $\bar{Z}_{\alpha}$ grows superlinearly as $\alpha \rightarrow \infty$ almost surely.
\end{theorem}
\begin{proof}
As \ref{thm:edge}.
\end{proof}

\subsection{Posterior inference}
We first characterise the posterior of $\mu_\alpha$ given $\mu'_\beta$ and $D_{\alpha\beta}$. The conditional Laplace functional of $\mu_\alpha$ given $D_{\alpha\beta}$ is $\mathbb{E}[e^{-\mu_\alpha(f)}|\mu'_\beta, D_{\alpha\beta}]$, for any non-negative measurable function $f$ such that $\mu_\alpha(f) = \sum_{i=1}^{\infty}w_i f(\theta_i)$. We have $\mu_\alpha(f) = \Pi(\tilde{f})$ where $\Pi = \sum_{i=1}^{\infty} \delta_{w_i, \theta_i}$ is a Poisson random measure on $\mathcal{S} = (0, \infty) \times [0, \alpha]$ with mean measure $\rho \times \lambda$ and $\tilde{f}(w, \theta) = wf(\theta)$. Let $n_{i**} = \sum_{j=1}^{N_\alpha}\sum_{k=1}^{N_\beta}n_{ijk}$, $m_i = \sum_{j=1}^{N_\alpha}\sum_{k=1}^{N_\beta} n_{ijk} + n_{jik}$, and $m'_k = \sum_{i=1}^{N_\alpha} \sum_{j=1}^{N_\alpha} n_{ijk}$.

\begin{align}
\label{eqn:lpl_d}
\mathbb{E}_{\mu_\alpha}[e^{-\mu_\alpha(f)}|D_{\alpha\beta}, \mu'_\beta] 
&= \mathbb{E}_{\Pi}[e^{-\int \tilde{f}(w, \theta)\Pi(dw, d\theta)}|D_{\alpha\beta}, \mu'_\beta] \\
&= \frac{\mathbb{E}_{\Pi}[ e^{-\Pi(\tilde{f})} P(D_{\alpha\beta}|\Pi, \mu'_\beta)]}{\mathbb{E}_{\Pi}[P(D_{\alpha\beta}|\Pi, \mu'_\beta)]}\\
%&= \frac{\mathbb{E}_{\Pi}[e^{-\Pi(\tilde{f})} e^{-\Pi(h)^2{\mu'}_\beta^{*}} \prod_{i=1}^{N_\alpha} w_i^{m_i} \prod_{k=1}^{N_\beta} {w'_k}^{m_k} ]}{\mathbb{E}_{\Pi}[e^{-\Pi(h)^2{\mu'}_\beta^{*}} \prod_{i=1}^{N_\alpha} w_i^{m_i} \prod_{k=1}^{N_\beta} {w'_k}^{m_k} ]]}
&= \frac{\mathbb{E}_{\Pi}[e^{-\Pi(\tilde{f})} e^{-\Pi(h)^2{\mu'}_\beta^{*}} \prod_{i=1}^{N_\alpha} w_i^{m_i}]}{\mathbb{E}_{\Pi}[e^{-\Pi(h)^2{\mu'}_\beta^{*}} \prod_{i=1}^{N_\alpha} w_i^{m_i}]]}
\end{align}
where $h(w, \theta) = w$ and
\begin{align}
P(D_{\alpha\beta}|\Pi, \mu'_\beta) & = P(D_{\alpha\beta}|\mu_\alpha, \mu'_\beta)\\
& = \text{Poisson}(D^*_{\alpha\beta}|\mu_\alpha^{*2}{\mu'}_\beta^{*}) 
\prod_{i=1}^{N_\alpha} P(n_{i**}|\mu_\alpha) \prod_{j=1}^{N_\alpha} P(n_{*j*}|\mu_\alpha)
\prod_{k=1}^{N_\beta} P(n_{**k}|\mu_\beta) \\
& = \frac{ (\mu_\alpha^{*2}{\mu'}_\beta^{*})^{D^*_{\alpha\beta}} e^{-\mu_\alpha^{*2}{\mu'}_\beta^{*}} }{D^*_{\alpha\beta}!}
\prod_{i=1}^{N_\alpha} \Big( \frac{w_i}{\mu_\alpha^*} \Big)^{n_{i**}} 
\prod_{j=1}^{N_\alpha} \Big( \frac{w_j}{\mu_\alpha^*} \Big)^{n_{*j*}} 
\prod_{k=1}^{N_\beta} \Big( \frac{w'_k}{{\mu'}_\beta^{*}} \Big)^{n_{**k}}\\
& =\frac{e^{-\mu_\alpha^{*2}{\mu'}_\beta^{*}} }{D^*_{\alpha\beta}!}
\prod_{i=1}^{N_\alpha} w_i^{m_i}
\prod_{k=1}^{N_\beta} {w'_k}^{m_k} 
 =\frac{e^{-\Pi(h)^2{\mu'}_\beta^{*}} }{D^*_{\alpha\beta}!}
\prod_{i=1}^{N_\alpha} w_i^{m_i}
\prod_{k=1}^{N_\beta} {w'_k}^{m_k}\\
\\
\mu_\alpha^* & = \sum_{i=1}^{\infty} w_i, \qquad {\mu'}_\beta^{*} = \sum_{k=1}^{\infty} w'_k = \sum_{k=1}^{N_\beta} w'_k + {w'}^*
\end{align}
Applying the generalised Palm formula to the numerator yields
\begin{align}
&\mathbb{E}_{\Pi}\Big[e^{-\Pi(\tilde{f})} e^{-\Pi(h)^2{\mu'}_\beta^{*}} \prod_{i=1}^{N_\alpha} w_i^{m_i} \Big] &\\
&= \mathbb{E}_{\Pi}\Big[e^{-\Pi(\tilde{f})} e^{-\Pi(h)^2{\mu'}_\beta^{*}} \prod_{i=1}^{N_\alpha} \sum_{w_j, \vartheta_j \in \Pi} w_j^{m_i} \mathbf{1}_{\theta_i}(\vartheta_j)\Big]& \\
&= \mathbb{E}_{\Pi}\Big[\int_{\mathcal{S}^{N_\alpha}} e^{-\Pi(\tilde{f})} e^{-\Pi(h)^2{\mu'}_\beta^{*}}  \prod_{i=1}^{N_\alpha} w_j^{m_i} \mathbf{1}_{\theta_i}(\vartheta_j) \Pi(dw_j, d\vartheta_j)\Big]& \\
&=  \int_{\mathcal{S}^{N_\alpha}} \mathbb{E}_{\Pi}\Big[ e^{-(\Pi+\sum_i^{N_\alpha}\delta_{(w_i, \theta_i)})(\tilde{f})} e^{-(\Pi+\sum_i^{N_\alpha}\delta_{(w_i, \theta_i)})(h)^2{\mu'}_\beta^{*}}\Big] \prod_{i=1}^{N_\alpha} w_j^{m_i} \mathbf{1}_{\theta_i}(\vartheta_j) \rho(dw_j)\lambda(d\vartheta_j) & \\
&=  \int_{\mathcal{S}^{N_\alpha}} \mathbb{E}_{\mu_\alpha}\Big[ e^{-\mu_\alpha(f) - \sum_{i=1}^{N_\alpha}w_if(\vartheta_j)} e^{-(\mu_\alpha(1) + \sum_{i=1}^{N_\alpha} w_i)^2{\mu'}_\beta^{*}}\Big] \prod_{i=1}^{N_\alpha} w_j^{m_i} \mathbf{1}_{\theta_i}(\vartheta_j) \rho(dw_j)\lambda(d\vartheta_j) &\\
&=  \int_{\mathcal{S}^{N_\alpha}} \mathbb{E}_{\mu_\alpha^*}\bigg[ \mathbb{E}_{\mu_\alpha} \Big[e^{-\mu_\alpha(f)}|\mu_\alpha^* \Big] e^{- \sum_{i=1}^{N_\alpha}w_if(\vartheta_j)} e^{-(\mu_\alpha^* + \sum_{i=1}^{N_\alpha} w_i)^2{\mu'}_\beta^{*}}\bigg] \prod_{i=1}^{N_\alpha} w_j^{m_i} \mathbf{1}_{\theta_i}(\vartheta_j) \rho(dw_j)\lambda(d\vartheta_j) &
\end{align}
The denominator is obtained by taking $f=0$.
\begin{align}
\mathbb{E}_{\mu_\alpha}[e^{-\mu_\alpha(f)}|D_{\alpha\beta}, \mu'_\beta] &= \int_{\mathbb{R}^{N_\alpha + 1}} E_{\mu_\alpha}[e^{-\mu_\alpha(f)}|\mu_\alpha^* = w^*]\\
&\quad \times e^{\sum_{i=1}^{N_\alpha}w_i f(\theta_i)} p(w_1, ..., w_{N_\alpha}, w^*|D_{\alpha\beta}, \mu_\beta) dw_{1:N_\alpha}dw^*
\end{align}
where
\begin{align}
p(w_1, ..., w_{N_\alpha}, w^*|D_{\alpha\beta}, \mu_\beta)&
= \frac{\prod_{i=1}^{N_\alpha} w_j^{m_i} \rho(w_i) e^{-(w^* + \sum_{i=1}^{N_\alpha} w_i)^2{\mu'}_\beta^{*}} g^*_\alpha(w^*)}
{\int_{\mathbb{R}^{N_\alpha + 1}}\prod_{i=1}^{N_\alpha} \tilde{w}_j^{m_i} \rho(\tilde{w}_i) e^{-(\tilde{w}^* + \sum_{i=1}^{N_\alpha} \tilde{w}_i)^2{\mu'}_\beta^{*}} g^*_\alpha(\tilde{w}^*) d\tilde{w}_{1:N_\alpha}d\tilde{w}^*}\\
\end{align}
$g^*_\alpha(w^*)$ is a density function of random variable $w^*$ of which Laplace transform is $\mathbb{E}[e^{tw^*}] = e^{\alpha\psi(t)}$. Therefore, the conditional of $\mu_\alpha$ given $D_{\alpha\beta}, \mu'_\beta$ is
\begin{align}
w^* \sum_{i=1}^{\infty}\tilde{P}_i\delta_{\tilde\theta_i} + \sum_{i=1}^{N_\alpha} w_i \delta_{\theta_i} 
\end{align}
where $(\tilde{P})$ are distributed from a Poisson-Kingman distribution conditional on $w^*$, and the weights $w_1, ..., w_{N_\alpha}, w^*$ are jointly dependent conditional on $D_{\alpha\beta}$ and $\mu'_\beta$:
\begin{align}
p(w_1, ..., w_{N_\alpha}, w^* | D_{\alpha\beta}, \mu'_\beta) \propto \prod_{i=1}^{N_\alpha}{w_i}^{m_i} e^{(-w_* + \sum_{i=1}^{N_\alpha}w_i)^2 {\mu'}_\beta^*} \prod_{i=1}^{N_\alpha} \rho(w_i) g^*_\alpha(w^*)
\end{align}

The conditional Laplace functional of $\mu'_\beta$ given $\mu_\alpha$ and $D_{\alpha\beta}$ can be carried out in the same way as we've done in $\mu_\alpha$.

\bibliographystyle{apalike}
\bibliography{ref}

\end{document}
